\section{SCU Modules}
\section{SCU Modules}
\section{SCU Modules}
\section{SCU Modules}
\input{fig/avr/modules} 

The AVR is responsible for controlling the entire system, and has several
important tasks and functionalities related to this. These have been split
into several modules that behave largely independently of each other.
\begin{description}
% \item[SD over SPI] We decided early on to use \ac{SD} cards as our main source
% of data. These are easy to communicate with, and easy to use framework code
% that we could build on already exists.  Even though \ac{SD} cards typically
% support several protocols of communication, our specific AVR's lack of
% hardware support for the 4-bit SD transfer mode left us with \ac{SPI} as our
% only choice. While \ac{SPI} is simple to use, it is also fairly slow. Our
% performance issues with \ac{SPI} is detailed further in section
% \ref{sec:avr-spi-issues}.

\item[SD over SPI] A simple to use source of data/data storage, with easy to use
  framework code available. It is rather slow, however, as the AT32UC3A0 only
  has support for SPI. Section \ref{sec:avr-spi-issues} covers this in detail.

% \item[FAT] Managing storage of multiple data sources ourselves on top of a
% block device such as \ac{SD} is difficult. A file system makes it a lot easier
% to manage several data sources simultaneously. However, it also incurs a
% performance penalty. Therefore we've chosen to implement what is not
% performance critical with \ac{FAT}.

\item[FAT] FAT is a file system implementation with easy to use framework
  code available. However, using it also incurs a performance penalty and is not
  used in performance critical sections. This is detailed in section
  \ref{sec:performance-sd-card}.

% \item[LENA/GPIO] As explained further in section
% \ref{sec:avr-LENA-communication}, we will communicate with the \ac{LENA} over
% a 8/24-bit interface. On the AVR, \ac{GPIO} is the subsystem responsible for
% reading and writing the pins on this bus.  On top of \ac{GPIO}, we wrote our
% code to abstract some of the details of \ac{GPIO}, and provide a more natural
% interface for us to use.

\item[LENA/GPIO] Interface for communicating with \ac{LENA}. It is built on
  top of GPIO and provides a natural interface without low level details of GPIO.

% \item[RS232] RS232 is our protocol of choice for serial communication with a
% PC. It was chosen for its simplicity, on both the AVR and PC side. It was
% never intended to transfer large amounts of data, but rather be a simple
% interface for us to get debugging information out of the AVR.

\item[RS232] A protocol for communication with a PC over the serial port. Very
  simple to use and is very useful for debugging. We did, however, not use it
  past the EVK1100 test board as we had an issue with it (See Section
  \ref{sec:avr-serial-issues}), nor did we really need it.

\TODO{her stod SMC/EBI før, er det egentlig en modul?} 
%\item[SMC/EBI] The AVR uses an External Bus Interface to connect to external
%  memories, both SRAM and \ac{SDRAM}. As our AVR has 512K of SRAM
%  attached, the EBI is our way of communicating with it. The \ac{SMC} is a
%  part of the EBI that provides us with a nice, memory-mapped interface to
%  our external SRAM.

% \item[BMP] In case we want to read standard bitmap files from \ac{SD}, we need
% a way to extract the raw pixel values from these files and send them in the
% correct order to the \ac{LENA} (\ac{BMP} files store images bottom-up, not
% top-down). This library takes care of this task. Deprecated since we decided
% to implement the actual image reading as reading raw data without even a file
% system.

\item[Str2Img] Str2Img is a small text-to-image library that we use to render
  text on the monitor. Primary use is debugging and creating the menu the users
  are presented with when the SCU resets.
  
\item[Program] This is responsible for presenting the user with an interface to
  select the program to run. It is built on top of other utility libraries such
  as Str2Img and our \ac{LENA} abstraction layer.

% Button & The user navigates the menu and controls the AVR using buttons. This
% is a thin simplification layer built on top of \ac{GPIO} that handles the
% interrupts and button click callbacks. \CHECK{this does not exists as of right
% now, but it might in the future :)}

\item[LEDs] Functions for turning the LED on and off. Useful for blinking
  lights to show information about internal states of the SCU and debugging.
  
\item[JTAG] JTAG is an interface used to connect the SCU to a
  computer. This allows us to flash the AVR, debug and profile our programs
  using familiar-looking tools: avr32program, avr32-gdb and avr32-gprof, for
  flashing, debugging and profiling, respectively.
\end{description}
 

The AVR is responsible for controlling the entire system, and has several
important tasks and functionalities related to this. These have been split
into several modules that behave largely independently of each other.
\begin{description}
% \item[SD over SPI] We decided early on to use \ac{SD} cards as our main source
% of data. These are easy to communicate with, and easy to use framework code
% that we could build on already exists.  Even though \ac{SD} cards typically
% support several protocols of communication, our specific AVR's lack of
% hardware support for the 4-bit SD transfer mode left us with \ac{SPI} as our
% only choice. While \ac{SPI} is simple to use, it is also fairly slow. Our
% performance issues with \ac{SPI} is detailed further in section
% \ref{sec:avr-spi-issues}.

\item[SD over SPI] A simple to use source of data/data storage, with easy to use
  framework code available. It is rather slow, however, as the AT32UC3A0 only
  has support for SPI. Section \ref{sec:avr-spi-issues} covers this in detail.

% \item[FAT] Managing storage of multiple data sources ourselves on top of a
% block device such as \ac{SD} is difficult. A file system makes it a lot easier
% to manage several data sources simultaneously. However, it also incurs a
% performance penalty. Therefore we've chosen to implement what is not
% performance critical with \ac{FAT}.

\item[FAT] FAT is a file system implementation with easy to use framework
  code available. However, using it also incurs a performance penalty and is not
  used in performance critical sections. This is detailed in section
  \ref{sec:performance-sd-card}.

% \item[LENA/GPIO] As explained further in section
% \ref{sec:avr-LENA-communication}, we will communicate with the \ac{LENA} over
% a 8/24-bit interface. On the AVR, \ac{GPIO} is the subsystem responsible for
% reading and writing the pins on this bus.  On top of \ac{GPIO}, we wrote our
% code to abstract some of the details of \ac{GPIO}, and provide a more natural
% interface for us to use.

\item[LENA/GPIO] Interface for communicating with \ac{LENA}. It is built on
  top of GPIO and provides a natural interface without low level details of GPIO.

% \item[RS232] RS232 is our protocol of choice for serial communication with a
% PC. It was chosen for its simplicity, on both the AVR and PC side. It was
% never intended to transfer large amounts of data, but rather be a simple
% interface for us to get debugging information out of the AVR.

\item[RS232] A protocol for communication with a PC over the serial port. Very
  simple to use and is very useful for debugging. We did, however, not use it
  past the EVK1100 test board as we had an issue with it (See Section
  \ref{sec:avr-serial-issues}), nor did we really need it.

\TODO{her stod SMC/EBI før, er det egentlig en modul?} 
%\item[SMC/EBI] The AVR uses an External Bus Interface to connect to external
%  memories, both SRAM and \ac{SDRAM}. As our AVR has 512K of SRAM
%  attached, the EBI is our way of communicating with it. The \ac{SMC} is a
%  part of the EBI that provides us with a nice, memory-mapped interface to
%  our external SRAM.

% \item[BMP] In case we want to read standard bitmap files from \ac{SD}, we need
% a way to extract the raw pixel values from these files and send them in the
% correct order to the \ac{LENA} (\ac{BMP} files store images bottom-up, not
% top-down). This library takes care of this task. Deprecated since we decided
% to implement the actual image reading as reading raw data without even a file
% system.

\item[Str2Img] Str2Img is a small text-to-image library that we use to render
  text on the monitor. Primary use is debugging and creating the menu the users
  are presented with when the SCU resets.
  
\item[Program] This is responsible for presenting the user with an interface to
  select the program to run. It is built on top of other utility libraries such
  as Str2Img and our \ac{LENA} abstraction layer.

% Button & The user navigates the menu and controls the AVR using buttons. This
% is a thin simplification layer built on top of \ac{GPIO} that handles the
% interrupts and button click callbacks. \CHECK{this does not exists as of right
% now, but it might in the future :)}

\item[LEDs] Functions for turning the LED on and off. Useful for blinking
  lights to show information about internal states of the SCU and debugging.
  
\item[JTAG] JTAG is an interface used to connect the SCU to a
  computer. This allows us to flash the AVR, debug and profile our programs
  using familiar-looking tools: avr32program, avr32-gdb and avr32-gprof, for
  flashing, debugging and profiling, respectively.
\end{description}
 

The AVR is responsible for controlling the entire system, and has several
important tasks and functionalities related to this. These have been split
into several modules that behave largely independently of each other.
\begin{description}
% \item[SD over SPI] We decided early on to use \ac{SD} cards as our main source
% of data. These are easy to communicate with, and easy to use framework code
% that we could build on already exists.  Even though \ac{SD} cards typically
% support several protocols of communication, our specific AVR's lack of
% hardware support for the 4-bit SD transfer mode left us with \ac{SPI} as our
% only choice. While \ac{SPI} is simple to use, it is also fairly slow. Our
% performance issues with \ac{SPI} is detailed further in section
% \ref{sec:avr-spi-issues}.

\item[SD over SPI] A simple to use source of data/data storage, with easy to use
  framework code available. It is rather slow, however, as the AT32UC3A0 only
  has support for SPI. Section \ref{sec:avr-spi-issues} covers this in detail.

% \item[FAT] Managing storage of multiple data sources ourselves on top of a
% block device such as \ac{SD} is difficult. A file system makes it a lot easier
% to manage several data sources simultaneously. However, it also incurs a
% performance penalty. Therefore we've chosen to implement what is not
% performance critical with \ac{FAT}.

\item[FAT] FAT is a file system implementation with easy to use framework
  code available. However, using it also incurs a performance penalty and is not
  used in performance critical sections. This is detailed in section
  \ref{sec:performance-sd-card}.

% \item[LENA/GPIO] As explained further in section
% \ref{sec:avr-LENA-communication}, we will communicate with the \ac{LENA} over
% a 8/24-bit interface. On the AVR, \ac{GPIO} is the subsystem responsible for
% reading and writing the pins on this bus.  On top of \ac{GPIO}, we wrote our
% code to abstract some of the details of \ac{GPIO}, and provide a more natural
% interface for us to use.

\item[LENA/GPIO] Interface for communicating with \ac{LENA}. It is built on
  top of GPIO and provides a natural interface without low level details of GPIO.

% \item[RS232] RS232 is our protocol of choice for serial communication with a
% PC. It was chosen for its simplicity, on both the AVR and PC side. It was
% never intended to transfer large amounts of data, but rather be a simple
% interface for us to get debugging information out of the AVR.

\item[RS232] A protocol for communication with a PC over the serial port. Very
  simple to use and is very useful for debugging. We did, however, not use it
  past the EVK1100 test board as we had an issue with it (See Section
  \ref{sec:avr-serial-issues}), nor did we really need it.

\TODO{her stod SMC/EBI før, er det egentlig en modul?} 
%\item[SMC/EBI] The AVR uses an External Bus Interface to connect to external
%  memories, both SRAM and \ac{SDRAM}. As our AVR has 512K of SRAM
%  attached, the EBI is our way of communicating with it. The \ac{SMC} is a
%  part of the EBI that provides us with a nice, memory-mapped interface to
%  our external SRAM.

% \item[BMP] In case we want to read standard bitmap files from \ac{SD}, we need
% a way to extract the raw pixel values from these files and send them in the
% correct order to the \ac{LENA} (\ac{BMP} files store images bottom-up, not
% top-down). This library takes care of this task. Deprecated since we decided
% to implement the actual image reading as reading raw data without even a file
% system.

\item[Str2Img] Str2Img is a small text-to-image library that we use to render
  text on the monitor. Primary use is debugging and creating the menu the users
  are presented with when the SCU resets.
  
\item[Program] This is responsible for presenting the user with an interface to
  select the program to run. It is built on top of other utility libraries such
  as Str2Img and our \ac{LENA} abstraction layer.

% Button & The user navigates the menu and controls the AVR using buttons. This
% is a thin simplification layer built on top of \ac{GPIO} that handles the
% interrupts and button click callbacks. \CHECK{this does not exists as of right
% now, but it might in the future :)}

\item[LEDs] Functions for turning the LED on and off. Useful for blinking
  lights to show information about internal states of the SCU and debugging.
  
\item[JTAG] JTAG is an interface used to connect the SCU to a
  computer. This allows us to flash the AVR, debug and profile our programs
  using familiar-looking tools: avr32program, avr32-gdb and avr32-gprof, for
  flashing, debugging and profiling, respectively.
\end{description}
 

The AVR is responsible for controlling the entire system, and has several
important tasks and functionalities related to this. These have been split
into several modules that behave largely independently of each other.
\begin{description}
% \item[SD over SPI] We decided early on to use \ac{SD} cards as our main source
% of data. These are easy to communicate with, and easy to use framework code
% that we could build on already exists.  Even though \ac{SD} cards typically
% support several protocols of communication, our specific AVR's lack of
% hardware support for the 4-bit SD transfer mode left us with \ac{SPI} as our
% only choice. While \ac{SPI} is simple to use, it is also fairly slow. Our
% performance issues with \ac{SPI} is detailed further in section
% \ref{sec:avr-spi-issues}.

\item[SD over SPI] A simple to use source of data/data storage, with
	easy to use framework code available. It is rather slow, however, as
	the AT32UC3A0 only has support for SPI, we had to use is. Section
	\ref{sec:avr-spi-issues} covers this in detail.

% \item[FAT] Managing storage of multiple data sources ourselves on top of a
% block device such as \ac{SD} is difficult. A file system makes it a lot easier
% to manage several data sources simultaneously. However, it also incurs a
% performance penalty. Therefore we've chosen to implement what is not
% performance critical with \ac{FAT}.

\item[FAT] FAT is a file system implementation with easy to use framework
  code available. However, using it also incurs a performance penalty and is not
  used in performance critical sections. This is detailed in Section
  \ref{sec:performance-sd-card}.

% \item[LENA/GPIO] As explained further in section
% \ref{sec:avr-LENA-communication}, we will communicate with the \ac{LENA} over
% a 8/24-bit interface. On the AVR, \ac{GPIO} is the subsystem responsible for
% reading and writing the pins on this bus.  On top of \ac{GPIO}, we wrote our
% code to abstract some of the details of \ac{GPIO}, and provide a more natural
% interface for us to use.

\item[LENA/GPIO] Interface for communicating with \ac{LENA}. It is built on
  top of GPIO and provides a natural interface without low level details of GPIO.

% \item[RS232] RS232 is our protocol of choice for serial communication with a
% PC. It was chosen for its simplicity, on both the AVR and PC side. It was
% never intended to transfer large amounts of data, but rather be a simple
% interface for us to get debugging information out of the AVR.

\item[RS232] A protocol for communication with a PC over the serial port. Very
  simple to use and is very useful for debugging. We did, however, not use it
  past the EVK1100 test board as we had an issue with it (See Section
  \ref{sec:avr-serial-issues}), nor did we really need it.

%\TODO{her stod SMC/EBI før, er det egentlig en modul?} 
% Nope, it's not.
%\item[SMC/EBI] The AVR uses an External Bus Interface to connect to external
%  memories, both SRAM and \ac{SDRAM}. As our AVR has 512k of SRAM
%  attached, the EBI is our way of communicating with it. The \ac{SMC} is a
%  part of the EBI that provides us with a nice, memory-mapped interface to
%  our external SRAM.

% \item[BMP] In case we want to read standard bitmap files from \ac{SD}, we need
% a way to extract the raw pixel values from these files and send them in the
% correct order to the \ac{LENA} (\ac{BMP} files store images bottom-up, not
% top-down). This library takes care of this task. Deprecated since we decided
% to implement the actual image reading as reading raw data without even a file
% system.

\item[Str2Img] Str2Img is a small text-to-image library that we use to render
  text on the monitor. Primary use is debugging and creating the menu the users
  are presented with when the SCU resets.
  
\item[Program] This is responsible for presenting the user with an interface to
  select the program to run. It is built on top of other utility libraries such
  as Str2Img and our \ac{LENA} abstraction layer.

% Button & The user navigates the menu and controls the AVR using buttons. This
% is a thin simplification layer built on top of \ac{GPIO} that handles the
% interrupts and button click callbacks. \CHECK{this does not exists as of right
% now, but it might in the future :)}

\item[LEDs] Functions for turning the LED on and off. Useful for blinking
  lights to show information about internal states of the SCU and debugging.
  
\item[JTAG] JTAG is an interface used to connect the SCU to a
  computer. This allows us to flash the AVR, debug and profile our programs
  using familiar-looking tools: avr32program, avr32-gdb and avr32-gprof, for
  flashing, debugging and profiling, respectively.
\end{description}
