\section{Structure of the Report}

This report may be seen as split up in three different parts: chapters
\ref{ch:intro}, \ref{ch:sys-over}-\ref{ch:pcb} and
\ref{ch:sys-test}-\ref{ch:conc}. \CHECK{Is the chapter numbering correct?} The
current chapter has given you an introduction to the assignment, our goals and
our team. \CHECK{Is the background chapter included?}
% Chapter \ref{ch:back} contains relevant background information for
% reading the rest of the report.

Chapters \ref{ch:sys-over} to \ref{ch:pcb} explains in deeper detail {\em how}
the machine works. Chapter \ref{ch:sys-over} gives an overview of the whole
machine and how the different parts are connected together. Chapter
\ref{ch:fpga} elaborates on the \ac{LENA} Architecture: How the \ac{FPGA} parts
of the machine is designed. \ac{SIMD} nodes, control core as well as the
\ac{FPGA} \ac{VGA} module is explained in great detail, along with the
interconnection between these. Chapter \ref{ch:avr} describes the software and
structures for the AVR, such as the file format and the program and data
menu. Chapter \ref{ch:pcb} explains the work around designing the \ac{PCB}, as
well as the soldering and other workarounds we had to do.

Chapters \ref{ch:sys-test} to \ref{ch:conc} is the conclusion part of the
report. Here we test the machine, discuss our choices and present our
conclusion, as well as possible further work. Chapter \ref{ch:sys-test} contains
tests to check whether we've managed to reach our requirements or not, along
with their results. Chapter \ref{ch:res} describes the result, performance and
energy consumption of the machine with different cores and different
programs. Chapter \ref{ch:disc} discusses different choices we did in our
project, and the result of these. In chapter \ref{ch:conc}, our conclusion is
given along with further work.
