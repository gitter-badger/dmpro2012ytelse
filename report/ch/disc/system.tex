\section{System design}
\begin{comment}
Choices made:\TODO{Unfinished braindump}
\begin{enumerate}
\item word width
\item instruction width/partitioning (different types of instructions?)
\item \ac{VGA} color switching in hardware
\end{enumerate}
\end{comment}

\subsection{Word size}

The word size of the \ac{LENA} architecture is an important aspect of the system which
has a major impact on performance. Our choice between small and large
register\TODO{Word size? But shouldn't confuse the reader between the size of
  the registers on the SIMD nodes and the instruction size either} size, and
small and large instruction sets will have a huge impact on the number of cores
we are able to get on the \ac{FPGA}. On one side, we have the Massively Parallel
Processor\cite{potter1985mpp} with 1 bit per core and focuses on many cores to
speed up the system. At the other side, we have Intel's MMX instructions which
allows one to do multiple, single instructions on different 64-bit and 128-bit
registers and is used to speed up e.g. image processing\cite{lee2004h264}. As
both techniques certainly work, we have to find the one which gives us the most
performance and is realizable within the resource constraints. Certainly, we
need to have multiple cores to fulfill NFR4, but the overall goal is to get
performance. If we get a better performance by allowing more operations in the
\ac{SIMD} nodes, we should do that instead.

From calculations\TODO{Should we have the calculations? JA. Jahre ønsker at det skal
være med en figur som viser ytelse som funksjon av ord-størrelse.} and an educated guess
on how complex such a system would be, we chose 8 bit register size and 24 bit instruction
size. From our assumptions, we could get around 16 \ac{SIMD} nodes and an
instruction set with reasonably efficient instructions for image processing.
\TODO{Jahre skrev at dette var muligens vårt  viktigste designvalg og der burde
skrivers mer om motivasjon, "motiver!".}
