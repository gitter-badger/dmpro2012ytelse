The \acf{SCU} is implemented on a AT32UC3A0512 AVR microcontroller. This was both an advantage and a disadvantage to us. The same AVR had been used in several previous projects which allowed us to learn from their successes and failures to some degree. However, this specific AVR also gave us some issues, especially in regard to performance of the \ac{SD} card.

As the name implies, the \ac{SCU} is the controlling unit in
our system. The \ac{SCU} is in charge of setting up both \ac{LENA}'s instruction
and data memory. The \ac{SCU} can start \ac{LENA} by putting it in a run state.

The \ac{SCU} also implements the user interface for the project. With the help
of the \ac{VGA} implementation on the \ac{LENA}, it is able to display a menu to
the user which allows for selection of a program and data source from the \ac{SD}
card connected to the \ac{SCU}.
