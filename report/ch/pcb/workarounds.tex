\section {Problems and Workarounds}
\subsection{Serial port}
We found many pin mappings for the serial port on the web, which turned out to be
different form Festina lentes serial system. After some discussion, we chose the one on the web,
and figured that if anything did not work, we could just use our header to change the mapping.
Yet as we had not thought of the fact that the footprint for male and female serial ports have
mirrored numbering, thus we ended up with a male pin mapping and a female footprint. 
Obviously, this did not work very well, yet as the TXD pin is the only one we actually need, 
we used a female plug, and dragged a wire over the header pins so that the TXD data from SCU 
were mapped to the original RXD. This workaround made it possible to send data from the system 
and out to a computer, yet as we had no need for this option, our final card do not have the serial soldered on.
\subsection{Routing}
Routing was the first problem we confronted during the designing the \ac{PCB}
board, we had to manually route the design, because we got some warnings due to
the clearance of the board. We used the auto-routing function provided by
Altium, yet as it did not provide satisfying results, we tried to change some
parameters to make them better, without success. Finally, we realized that we
had to route a few wires manually. This was mostly motivated by our desire to
not have vias as close to the pins as Altium placed them. Also, as to cut down
the budget, we removed some unnecessary vias by wiring multiple grounds and
powers together, where the design allowed us to. In this way, we got quite a few
less vias.
\subsection{Soldering}
Soldering was the biggest problem that we faced, none of the \ac{PCB} group
member had any experiences with it beforehand. We managed to destroy our first
board, as we used too much tin on one of the AVR pins, and ruined them by using
the tin removal unit wrongly.

The next try, however, proved to be successful, but there still remained some
minor problems. Problems we found were lack of tin on the oscillator, as well as
a few of the \ac{FPGA} pins. They were, as opposed to our problem with the first
board, easily fixed, once discovered.


