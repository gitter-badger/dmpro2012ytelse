\pdfbookmark[1]{Abstract}{Abstract}
\begingroup
\let\clearpage\relax
\let\cleardoublepage\relax
\let\cleardoublepage\relax

\chapter*{Abstract}
% \TODO{This is a very rough draft written after 24 hours of sleeplessness. Be
% ruthless in your editing/rewriting/deleting/whatever.} As heat generation and
% power consumption has put up a ceiling for how large and fast processors can
% be built, parallelism has risen to become the de facto standard for computing.
% Everything from cellphones to desktop computers now carry multiple processing
% cores to perform their functions. But parallelism is of course not a new
% concept; for many decades exploiting the parallel nature of certain data to
% effectivize computation has been a known tactic. Especially within the field
% of image processing has parallelism proved itself a good fit.

% Although MIMD systems have taken hold as the dominant class of parallel
% systems for general computing, we wanted to explore alternatives which may
% work better for certain applications. The result of this project is the
% computer ``256 Shades of Gray``, an array based SIMD architecture optimized
% for very parallelizable tasks such as image processing.

% ``256 Shades of Gray''(henceforth refered to as ``TSOG``\TODO{Not AFAIK. We
% use ``256 Shades of Gray'' consistently}) consists of a custom PCB design, a
% microprocessor system control unit and a FPGA on which our SIMD architecture
% is implemented. It accepts input through a SD card reader (with RS-232 and USB
% as backup), and shows its output through a custom VGA controller.

% We have been greatly inspired by the Goodyear MPP architecture, which clearly
% shows in our design. Simplicity was strived for at all
% times\TODO{Well... Simplicity hasn't been our focus, performance has.} to be
% able to fit as many cores as possible within limited resources.

% In the end the design has gone through fairly rigorous testing, and has met
% both the functional and non-functional requirements we set out to meet. The
% architecture has shown itself a good fit for multiple common image processing
% tasks.
\TODO{Jeg liker ikke førstesetningen: "put up a limit" høres ikke bra ut. det 
begrenser heller ikke hvor fort prosessorer kan designes :)}
As heat generation and power consumption put up a limit of how large and fast a
single processor can be designed, parallelism is becoming more and more
important within all performance demanding computing. Everything from cellphones
to desktop computers now carry multiple processing cores to perform their
functions. Parallelism is of course not a new concept; for many decades
exploiting the parallel nature of certain data to effectivize computation has
been a known tactic. Especially within the field of image processing parallelism
has proved itself an extremely effective tool.

While MIMD systems are dominant within general computing, we want to explore
alternatives which may work better for certain applications. The result of this
project is the computer ``256 Shades of Gray'', which is
an array based SIMD architecture optimized for massively parallelizable tasks
such as image processing.

256 Shades of Gray consists of a custom PCB design, a microprocessor operating
as a System Control Unit and an FPGA which implements our SIMD architecture. It
takes input through an SD card reader (with RS-232 and USB as
backup\TODO{Nødvendig å kommentere hva som er backup?}), and shows output
through a custom VGA controller.

%We have been inspired by the Goodyear MPP architecture, which clearly shows in
%our design. We designed for simplicity as much as possible\TODO{Har vi designet
%mtp. simplicity? Vi har designed mtp. performance. Vi kunne lett unngått {\sc
%Store \& Forward}-greiene i SIMD nodene og fått en mye mindre kompleks sak.}, in
%order to be able to fit as many cores as possible within our limited
%resources. As such, the SIMD nodes have been stripped of all complex
%instruction, all floating point capabilities, integer multiplication and integer
%division. The SIMD nodes are all controlled by a control core that handles all
%memory accesses through a DMA unit.

%The SD card and the custom VGA controller both worked flawlessly in the
%end\TODO{``in the end'' skurrer litt}, and while the RS-232 port did not work at
%all the VGA could replace it for debugging purposes. In the end our design has
%gone through testing, and has met all functional and non-functional requirements
%that were set. The architecture has shown itself to be effective and well suited
%for common image processing tasks.

We have been inspired by the Goodyear MPP architecture, which clearly shows in
our design. Our main focus was performance, and because of that simplicity in
design has been strived for in order to fit as many parallel cores on our FPGA
as possible. As such, the SIMD nodes have been stripped of many of the more
complex instructions, such as integer multiplication and division, as well as
all floating point capabilities. The SIMD array is controlled by a control core
that handles I/O and memory access through a DMA unit.

Our design has gone through extensive testing, and has met all functional and
non-functional requirements that were set. The architecture has shown itself
effective and well suited for common image processing tasks.
\endgroup
