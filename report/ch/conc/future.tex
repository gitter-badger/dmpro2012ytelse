\section{Future Work}
The limited number of cores we were able to fit on the FPGA is a major limitation
of our system. Future work could be to work on reducing the complexity of the
SIMD nodes to fit more onto the FPGA, and another option may be to choose a larger
FPGA.

Once the core design has been improved to fit more cores, or a much larger FPGA is
used, we might consider adding some instructions we chose to strip out. Multiplication
and division are two very useful instruction we did not include that, if included, would simplify
general coding on the LENA architecture.

Another major limitation was the reading speed from the SD card.
The chosen microcontroller, the AT32UC3A0512, has limited possibilities of input
from mass storage devices. A possibly faster solution is Ethernet input, but this
would require another computer to feed information to the system so the group
as not looked into this possibilitiy.

Another possibility for improving input speed could be to switch to an AT32UC3A3xxx
microcontroller, as these have support for High Speed USB (USB 2.0), which should be
able to read much faster than our SD card reader.
\begin{itemize}
\item If we removed the special ``Get all 4 neighbor values at
  once''-instructions and other special parts of the system, would we be able to
  get more cores?
\item If we strip the \ac{SIMD} nodes almost to the bone, barely being Turing
  complete, how many nodes would we get?
\end{itemize}
