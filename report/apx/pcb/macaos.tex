\section{Macaos}\label{app:macaos}

This part is intended to provide a few insights about the ins and outs of using
Macaos for creating production-files, as we feel it might be useful to note down
what we learned for those that come after us.

\begin{enumerate}
\item Visit the Macaos homepage and apply for a license
  key\footnote{\url{http://www.macaos.com/products/me/get}}, remember to leave
  some time for a reply on this, as these requests seem to be handled manually.
\item In Altium Designer, use File-Fabrication Outputs-Gerber Files. See
  Macaos documentation for decent settings, take specific care regarding the
  Digit format, and zero-suppression settings.
\item In Altium Designer, use File-Fabrication Outputs-NC Drill Files. As above,
  see the Macaos documentation for proper settings, they are likely the same as
  in the Gerber Files.
\item The necessary files should now exist in your Project folder under
  ``Project Outputs for Projectname''.
\item Install and open Macaos. You will be asked for a license key, which you
  should have gotten along with the download link.
\item Click ``Import''.
\item At this point it might be useful to have file endings enabled in
  Explorer, so open an Explorer window, and go to Tools, Folder-Options, and
  remove the checkmark for ``Hide extensions for known file types''.
\item Click ``Open Files''.
\item Select the relevant files for import, crossreferencing against the
  Altium documentation might give a decent idea about which files go where.
  Importing all the files might be a good idea, as you might have an easier time
  selecting the correct files when watching the preview of each file.
\item Bind the files to their respective layers. Most of the layers should be
  easily spottable, like GTO for Notation Top, GTS for Top Solder Mask, GP1 and
  GP2 for internal layers 1 and 2. Take care to make sure these are not
  negative. If som then select Negative Layer 1/2 instead of Inner Layer 1/2.
  The Copper layers will likely be GTL and GBL. make sure these are correct
  too. Additional files include the Keepout (GKO), which can be used as
  ``Board''.
\item Verify that the drill files are properly selected, by clicking on ``Drill
  Tools'' in Macaos. You will probably have to link a text file ending in
  -RoundHoles.txt to ``Drill''. At this point it might also be a good idea to
  verify that the amount of holes match.
\item Verify that the list of layers in Macaos is complete.
\item Click ``Contour'' in Macaos, a Board contour will be autogenerated for
  you.
\item Click ``Board specs'', select a stackup. You would typically use 4001,
  which is listed as standard.
\item If wanted, you can select a different colour for your board/silk layer
  here as well.
\item Click ``Board stats'', check the hole amount, and set the ``Minimum
  feature sizes''. These values depend a bit on what settings you ended up with
  in Altium, but usually 0.125 and 0.15 mm seem to be decent values.
\item Click ``Place Product/Batch number'', and place that on your board.
  Alternatively skip this, and let Macaos pick a place for it.
\item Click ``Publish to product browser'', which allows you to get a price
  estimate on your board.
\item If you did all this with your own license, you might also want to click
  ``File, Save to local disk'', to generate a {\tt .mei}-file that you can
  deliver to whoever will be doing the actual ordering.
\end{enumerate}

Useful links:
\begin{itemize}
\item \url{http://www.macaos.com/products/me/get} | Applying for a
  Macaos-license
\item \url{http://www.macaos.com/support/gerber/other} | Generating Gerber-files
\item \url{http://wiki.altium.com/display/ADOH/Gerber+Output+Options} |
  Altium-information about generated Gerber-files.
\end{itemize}
