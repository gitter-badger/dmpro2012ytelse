\section{Component Functionality}

This section describes the different components and their functionality. A list
of ordered parts can be found in \ref{app:ordered-parts}.
\TODO{Jahre vil ha figur(er)}
\subsection{SCU}

The microcontroller is mainly used to send data from the \ac{SD} card to the
\ac{LENA} architecture. It also converts data from \ac{BMP} to a format
\ac{LENA} is able to process.\CHECK{Vi bruker ikke BMP lenger, så vi må
  omformulere dette!}  \ac{LENA} can be started and stopped by the \ac{SCU}, and
it is also responsible for sending the wanted program to \ac{LENA}, which stores
it in the instruction memory.

Selecting programs is done by the \ac{SCU}. This is done by listening on the
buttons we have on the machine.

\subsection{LENA}

NF4 constrained our high level choices on the image processor architecture:
There was a requirement that we had to have multiple cores arranged in a
matrix. As the matrix would perform image processing, it was natural for us to
choose a \ac{SIMD} architecture. Many image processing algorithms do the exact
same operation on every pixel, and having a \ac{SIMD} architecture reduces both
complexity and size needed per core on the \ac{FPGA} significantly.

Other design choices that followed was the introduction of a control core, a
\ac{DMA} and a \ac{VGA} controller. The control core is responsible for sending
data to the \ac{SIMD} nodes and the \ac{VGA} controller, whereas the \ac{DMA} is
responsible for writing data from the \ac{SIMD} nodes back into memory. The
\ac{VGA} controller is responsible for handling the \ac{VGA} memory and sending
the correct signals to the \ac{VGA} port. In addition, as the \ac{SIMD} nodes
usually depend heavily on their neighbour's data, we decided to have ``dummy
nodes'' outside the real \ac{SIMD} nodes. Their only function is to transmit
data to the edge nodes. As such, we can still utilize the edge nodes for
computation when neighbour data is needed.

\subsubsection{Word size}

The word size of the \ac{LENA} architecture is an important aspect of the system which
has a major impact on performance. Our choice between small and large
register\TODO{Word size? But shouldn't confuse the reader between the size of
  the registers on the SIMD nodes and the instruction size either} size, and
small and large instruction sets will have a huge impact on the number of cores
we are able to get on the \ac{FPGA}. On one side, we have the Massively Parallel
Processor\cite{potter1985mpp} with 1 bit per core and focuses on many cores to
speed up the system. At the other side, we have Intel's MMX instructions which
allows one to do multiple, single instructions on different 64-bit and 128-bit
registers and is used to speed up e.g. image processing\cite{lee2004h264}. As
both techniques certainly work, we have to find the one which gives us the most
performance and is realizable within the resource constraints. Certainly, we
need to have multiple cores to fulfil NFR4, but the overall goal is to get
performance. If we get a better performance by allowing more operations in the
\ac{SIMD} nodes, we should do that instead.

From calculations\TODO{Should we have the calculations? JA. Jahre ønsker at det skal
være med en figur som viser ytelse som funksjon av ord-størrelse.} and an educated guess
on how complex such a system would be, we chose 8 bit register size and 24 bit instruction
size. From our assumptions, we could get around 16 \ac{SIMD} nodes and an
instruction set with reasonably efficient instructions for image processing.
\TODO{Jahre skrev at dette var muligens vårt  viktigste designvalg og der burde
skrivers mer om motivasjon, "motiver!".}

\subsubsection{Memory}
\TODO{Should memory be here? I believe it is an important aspect of the whole
  machine, but maybe others disagree \ldots} The memory is split over three
different chips: The data memory, the instruction memory and the \ac{VGA}
memory. All of the \ac{RAM} is asynchronous \ac{SRAM}, based on previous
projects success with it\cite{berg2011festinalente} along with the speed. The
main reason for this choice was mostly due to frequent reading and writing, but
also because a larger \ac{SRAM} module was more costly than multiple smaller
\ac{SRAM} modules per \ac{MB}. An additional benefit from this is that the
complexity is lowered: Fewer potential collisions on \ac{RAM} reading and
writing occurs since fewer parts read and write to the different \ac{RAM} chips.

\subsubsection*{Instruction Memory}

We ended up with a 24-bit instruction set for both \ac{SIMD} nodes and the
control core. As we ended up with a 16-bit program counter, the maximal
instruction count would be 64,000 and the memory chip size is chosen on that
basis. While it may sound like few operations, especially considering how the
\ac{SIMD} nodes cannot jump or branch, it is still more than enough. To see why,
note that the control core can simulate loops and may start the image processing
algorithm from the beginning when the \ac{SIMD} nodes have finished processing
the current data.

\subsubsection*{Data Memory}

\subsubsection*{VGA Memory}

\subsection{I/O}

