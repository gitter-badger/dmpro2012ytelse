\section {Routing}

We spent the entire final week before production working on the routing,
the reason for the time span needed for this was a result of a couple of
things.

(TODO: Expand the following into text)
Why we used so much time: (Or, what the other group did right, and we did wrong)
\begin {itemize}
\item We were the first group to start routing, and thus got to fall into all the gotcha-traps that
exist for routing, with no-one to walk into them ahead of us.
\item We didn't think about physical proximity when laying out the pins
\item We had 5 memory chips, with all the extra connections that implies.
\item We didn't set limitations until a few days into the week.
\end {itemize}

Initially we attempted auto-routing, which took the better part of half an hour,
even before the constraints were set, this showed us quite a few issues that needed
to be handled (and even more so when we finally got the constraints in).

TODO: Fill in about the types of errors/warnings we got.
\begin{itemize}
\item Constraints violations set wrong
\item Power-plane net-labeled wrong.
\item Clearance constraints violated by Altium
\item Short-circuiting vias.
\item Overlapping vias.
\item In general, the auto-routing started to produce violations 
\item more...
\end{itemize}
TODO: Silk-over-silk in USB/Lenna, errors we could ignore.
TODO: Scaling of board.

In the end we remapped some pins to increase physical proximity, and to untangle the
amount of crossing wires. Although care was taken during this process, we accidentally
happened to disconnect one pin in the schematic while reordering, we did notice this before
production though, and were able to correct the mistake by manually routing the connection.
Finally we did some manual routing to reduce the amount of unneccessary vias. \\

\subsection {Lenna and Text}
In the top right corner of our board, we put the "Lenna"-picture, this image
was produced by doing an edge-detection-pass on the image, then putting it through
the image-converter-plugin in Altium: \url{http://wiki.altium.com/display/ADOH/How+to+import+a+graphic+onto+the+PCB+overlay}

We also put our names, and the name of the project on the board, this was simply done
with the Text-label tool in Altium's PCB-Designer. This didn't allow for norwegian letters,
but this was fixed easily by adding lines and circles manually (as well as moving a's and e's together
to form æ).
