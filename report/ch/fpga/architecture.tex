\section{Architecture}

\section{Architecture}

\section{Architecture}

\section{Architecture}

\input{fig/fpga/architecture}

% Mention clock rate and possibly the reason for choosing it somewhere

An overview of the architecture of LENA as well as its memory and communication
lines is shown in figure \ref{fig:fpga-architecture}. The data flows mostly
through it in one direction. Data is received from the AVR and stored to the
data memory. Then the data is read in parts from the data memory and sent
through the SIMD array for processing. The result from the SIMD array is stored
to the data memory again. At last, the data is copied from the data memory to
the VGA memory and shown on screen.

% Mention that the matrix is well suited for some image filters

The processing part of LENA is the SIMD array. It is organized in a matrix,
as specified by the assignment, and designed to process images. Each node
processes one word at a time and can communicate with its four direct
neighbours. To load the first data into the array, it is sent to the
first column
of nodes and copied to the right, passing through all the nodes until the last
column. This is done by using a special register for sending data, called the
S-register. Data is continously received by the first row, so the whole array is
filled. At that time, data processing can start. The data is swapped onto
another register of the node, and new data can be loaded through the S-register.
When both data processing and transmission of new data is done, the registers
are swapped again. At this point, we have results in the S-registers and can
send this out to the right while loading new data from the left. The results is
copied out of the array from the last row.

Data communication between AVR, the SIMD nodes and the VGA controller is handled
by the control module which communicates with the data- and program memory, and
keeps track of the data location on these. The control module sends instructions
to all of the SIMD nodes and is responsible for splitting the data and loading
it into the SIMD array in correct order by setting the S-registers of the first
column. It also retrieves the results from the last column of the array and stores it
to the data memory.

The FPGA receives data from the AVR on 24 input lines. This is enough to
transmit one instruction per tramsmission. All of the instructions are copied
onto the instruction memory before any other data is sent. The image data is
sent at one word (8 bits) at a time. We considered using all 24 lines for data
transmission, but as we tested the transmission rate, we realized that it was
fast enough by only trasmitting one word at a time. This was easier to
implement, and was therefore the chosen implementation.

The SIMD array is created dynamically in VHDL, hence the number of cores is
easily interchangable. The final number of cores in our implementation is only
limited by the available space on the FPGA. We tried to fit as many nodes as
possible, and the FPGA we used had enough space for 6 nodes.


% Mention clock rate and possibly the reason for choosing it somewhere

An overview of the architecture of LENA as well as its memory and communication
lines is shown in figure \ref{fig:fpga-architecture}. The data flows mostly
through it in one direction. Data is received from the AVR and stored to the
data memory. Then the data is read in parts from the data memory and sent
through the SIMD array for processing. The result from the SIMD array is stored
to the data memory again. At last, the data is copied from the data memory to
the VGA memory and shown on screen.

% Mention that the matrix is well suited for some image filters

The processing part of LENA is the SIMD array. It is organized in a matrix,
as specified by the assignment, and designed to process images. Each node
processes one word at a time and can communicate with its four direct
neighbours. To load the first data into the array, it is sent to the
first column
of nodes and copied to the right, passing through all the nodes until the last
column. This is done by using a special register for sending data, called the
S-register. Data is continously received by the first row, so the whole array is
filled. At that time, data processing can start. The data is swapped onto
another register of the node, and new data can be loaded through the S-register.
When both data processing and transmission of new data is done, the registers
are swapped again. At this point, we have results in the S-registers and can
send this out to the right while loading new data from the left. The results is
copied out of the array from the last row.

Data communication between AVR, the SIMD nodes and the VGA controller is handled
by the control module which communicates with the data- and program memory, and
keeps track of the data location on these. The control module sends instructions
to all of the SIMD nodes and is responsible for splitting the data and loading
it into the SIMD array in correct order by setting the S-registers of the first
column. It also retrieves the results from the last column of the array and stores it
to the data memory.

The FPGA receives data from the AVR on 24 input lines. This is enough to
transmit one instruction per tramsmission. All of the instructions are copied
onto the instruction memory before any other data is sent. The image data is
sent at one word (8 bits) at a time. We considered using all 24 lines for data
transmission, but as we tested the transmission rate, we realized that it was
fast enough by only trasmitting one word at a time. This was easier to
implement, and was therefore the chosen implementation.

The SIMD array is created dynamically in VHDL, hence the number of cores is
easily interchangable. The final number of cores in our implementation is only
limited by the available space on the FPGA. We tried to fit as many nodes as
possible, and the FPGA we used had enough space for 6 nodes.


% Mention clock rate and possibly the reason for choosing it somewhere

An overview of the architecture of LENA as well as its memory and communication
lines is shown in figure \ref{fig:fpga-architecture}. The data flows mostly
through it in one direction. Data is received from the AVR and stored to the
data memory. Then the data is read in parts from the data memory and sent
through the SIMD array for processing. The result from the SIMD array is stored
to the data memory again. At last, the data is copied from the data memory to
the VGA memory and shown on screen.

% Mention that the matrix is well suited for some image filters

The processing part of LENA is the SIMD array. It is organized in a matrix,
as specified by the assignment, and designed to process images. Each node
processes one word at a time and can communicate with its four direct
neighbours. To load the first data into the array, it is sent to the
first column
of nodes and copied to the right, passing through all the nodes until the last
column. This is done by using a special register for sending data, called the
S-register. Data is continously received by the first row, so the whole array is
filled. At that time, data processing can start. The data is swapped onto
another register of the node, and new data can be loaded through the S-register.
When both data processing and transmission of new data is done, the registers
are swapped again. At this point, we have results in the S-registers and can
send this out to the right while loading new data from the left. The results is
copied out of the array from the last row.

Data communication between AVR, the SIMD nodes and the VGA controller is handled
by the control module which communicates with the data- and program memory, and
keeps track of the data location on these. The control module sends instructions
to all of the SIMD nodes and is responsible for splitting the data and loading
it into the SIMD array in correct order by setting the S-registers of the first
column. It also retrieves the results from the last column of the array and stores it
to the data memory.

The FPGA receives data from the AVR on 24 input lines. This is enough to
transmit one instruction per tramsmission. All of the instructions are copied
onto the instruction memory before any other data is sent. The image data is
sent at one word (8 bits) at a time. We considered using all 24 lines for data
transmission, but as we tested the transmission rate, we realized that it was
fast enough by only trasmitting one word at a time. This was easier to
implement, and was therefore the chosen implementation.

The SIMD array is created dynamically in VHDL, hence the number of cores is
easily interchangable. The final number of cores in our implementation is only
limited by the available space on the FPGA. We tried to fit as many nodes as
possible, and the FPGA we used had enough space for 6 nodes.


An overview of the architecture of \ac{LENA} as well as its memory and
communication lines is shown in figure \ref{fig:fpga-architecture}. The data
flows mostly through it in one direction. Data is received from the AVR and
stored to the data memory. The data is then read in parts from the data memory
and sent through the \ac{SIMD} array for processing. The result from the
\ac{SIMD} array is stored to the data memory again. \TODO {Say why the DMA 
doesn't send data to both data and VGA ram, yet have to use control core 
instead. (Supposedly because  the output image is more stable this way?
 )} At last, the data is copied
from the data memory to the \ac{VGA} memory and shown on screen.

The processing part of \ac{LENA} is the \ac{SIMD} array. It is organized
in a two-dimensional matrix, as specified by the assignment. Each node
processes one word at a time and can communicate with its four direct
neighbors. In order for the nodes to receive valid data from the edges
where they have no neighbors, special \emph{edge nodes}, that only hold
data values, are inserted along the edges.

These features make the architecture especially well suited for applying
3x3 image filters, such as noise reduction or edge detection filters.
See Appendix \ref{apx:laplace} for an example implementation of an edge
detection filter. \TODO{JN: Fix reference AFTER Minted-merge}

To first load the data into the \ac{SIMD} array, it is sent to the first column
of the nodes and copied to the right, passing through all the nodes until the
last column. This is done by using a special register for sending data, called
\NOTE{If this could be shortened somehow and fixed in the simd.tex-file, that
  would be great. However, it has low impact and requires MUCH work}
the S register. Data is continuously received by the first column, so the whole
array is filled. At that time, data processing can start. The data is swapped
onto another register of the node, and new data can be loaded through the S
register. When both data processing and transmission of new data is done, the
registers are swapped again. At this point, we have results in the S registers
and can send this out to the right while loading new data from the left. The
results are copied out of the array from the last column.

Data communication between \ac{SCU}, the \ac{SIMD} nodes and the \ac{VGA}
controller is handled by the control module which communicates with the data-
and program memory, and keeps track of the data location on these. The control
module sends instructions to all of the \ac{SIMD} nodes and is responsible for
splitting the data and loading it into the \ac{SIMD} array in correct order by
setting the S registers of the first column. It also retrieves the results from
the last column of the array and stores it to the data memory.

The \ac{FPGA} receives data from the AVR on 24 input lines. This is enough to
transmit one instruction per transmission. All of the instructions are copied
into the instruction memory before any other data is sent. The image data is
sent one word (8 bits) at a time. We considered using all 24 lines for data
transmission, but as we tested the transmission rate, we realized that it was
fast enough by only sending one word at a time. This was easier to
implement and was therefore the chosen implementation.

The \ac{SIMD} array is created dynamically in \ac{VHDL}, hence the
number of cores is easily changeable. The final number of cores in our
implementation is only limited by the available space on the \ac{FPGA},
and we have tried to fit as many nodes as possible. On the Spartan-3E
XC3S500E \ac{FPGA} we had enough space for 6 nodes, in addition to the
14 corresponding edge nodes, with a total slice utilization of 93 \%.
