\pdfbookmark[1]{Abstract}{Abstract}
\begingroup
\let\clearpage\relax
\let\cleardoublepage\relax
\let\cleardoublepage\relax

\chapter*{Abstract}
As heat generation and power consumption starts to limit speed and effectiveness
of single core architectures, parallelism is becoming more and more important
within all performance demanding computing. Everything from cell phones to
desktop computers now carries multiple processing cores to perform their
functions. Parallelism is of course not a new concept; for many decades,
exploiting the parallel nature of certain data to effectivize computation has
been a known tactic. Especially within the field of image processing,
parallelism has proved itself to be an extremely effective tool.

While \ac{MIMD} systems are dominant within general computing, we wanted to
explore alternatives which may work better for certain applications. The result
of this project is the computer ``256 Shades of Gray'', which is an array-based
\ac{SIMD} architecture optimized for massively parallelizable tasks such as
image processing.

256 Shades of Gray consists of a custom PCB design, a microprocessor operating
as a System Control Unit and an FPGA which implements our SIMD architecture. It
takes input through an SD card reader (with RS-232 and USB as backup), and shows
output through a custom VGA controller.

We have been inspired by the Goodyear MPP architecture, which clearly shows in
our design. Our main focus was performance, and because of that simplicity in
design has been strived for in order to fit as many parallel cores on our FPGA
as possible. As such, the SIMD nodes have been stripped of many
complex instructions, such as integer multiplication and division, as well as
all floating point capabilities. The SIMD array is controlled by a control core
that handles I/O and memory access through a DMA unit.

Our design has gone through testing, and has met all functional and
non-functional requirements that were set. The architecture has shown itself
effective and well suited for common image processing tasks.
\endgroup
