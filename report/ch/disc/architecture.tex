\section {Architectural design}

While still in the early phases of the project, multiple solutions as to how the data memory was suggested. The reason we decided to let the AVR send the FPGA all the data, and then let it store them to it's own local memory, was due to all the possible issues with shared memory. A shared memory would easily be prone to errors from synchronization problems for when to read and write, and to what addresses. Instead, we figured we'd just send all data from the AVR to the FPGA, and let it store them into it's own memory as it saw fit.\\

The program memory is separated on two chips, as we found no good 24 bit word memory, and instead combined two 16 bit ones.

\TODO {find out exactly what was the issue here, with the AVR}
As we got deeper into the techincal parts of our system, we figured we needed two additional memories. One for the AVR, as it had little on-chip memory, and thus might get some issues as it reads data from the SD card in blocks.(?) Another argument for adding this memory, was for our backup solution to work, as the FPGA would then have to send data back to the AVR, and the AVR would then have to store it somewhere.\\

Also, as a shared VGA and data memory might obstruct the flow of data from the AVR, as the data for the VGA would be both read from and written to a lot, the VGA needed it's own memory, rather than to share space with the data.\\

We feel that both of these additional memory components were necessary, even though we might have done without the AVR memory, not having it would significantly lowered the preformance if the VGA module had to be used.\\

\TODO {Find out if serial works}
As both the earlier groups had done so, and because Magnus told us to, we had multiple redundancies in order to safe guard against possible errors. In order to send data into the system, our two main options would be to use the SD card or mini USB plug. For debuging purposes, we had serial and JTAG, and for the output we had the two VGA options. Also, we used headers so as to be able to alter our pin usage, if need be. Luckily, both SD card and and our own VGA worked, and we therefore had no need of the redundancies.
