\section{Component Functionality}

This section describes the different components and their functionality. A list
of ordered parts can be found in \ref{app:ordered-parts}.

\subsection{System Control Unit}

The \acf{SCU} is used to control the \ac{LENA} architecture and as a user
interface.  The SCU sends data and instructions from the \ac{SD} card to
\ac{LENA}, which stores it in it's data and instruction memory,
respectively. The \ac{SCU} also starts and stops \ac{LENA}'s program.

Selecting programs and data is done by the user interface on the \ac{SCU} using
buttons as input and \ac{LENA}'s VGA as output.

\subsection{LENA}

NF4 constrained our high level choices on the image processor architecture:
There was a requirement that we had to have multiple cores arranged in a
matrix. As the matrix would perform image processing, it was natural for us to
choose a \ac{SIMD} architecture. Many image processing algorithms do the exact
same operation on every pixel, and having a \ac{SIMD} architecture reduces both
complexity and size needed per core on the \ac{FPGA} significantly.

Other design choices that followed was the introduction of a control core, a
\ac{DMA} and a \ac{VGA} controller. The control core is responsible for sending
data to the \ac{SIMD} nodes and the \ac{VGA} controller, whereas the \ac{DMA} is
responsible for writing data from the \ac{SIMD} nodes back into memory. The
\ac{VGA} controller is responsible for handling the \ac{VGA} memory and sending
the correct signals to the \ac{VGA} port. In addition, as the \ac{SIMD} nodes
usually depend heavily on their neighbor's data, we decided to have ``dummy
nodes'' outside the real \ac{SIMD} nodes. Their only function is to transmit
data to the edge nodes. As such, we can still utilize the edge nodes for
computation when neighbor data is needed.
