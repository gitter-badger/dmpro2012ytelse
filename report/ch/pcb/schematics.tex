\section {Schematics}

We decided to design the entire PCB in one schematic, because having read
the Festina Lente report, we noticed that they mention that using multiple schematics
might give some interesting issues (which we never got any further into,
as we never tried to use more than one schematic).

A major downside of this approach was the fact that this completely serialized our
work on the Schematic, since we couldn't make concurrent changes to our single document.
This limited the paralellism of our work-flow to one person working on the schematic,
and the other people in the PCB-group doing whatever could be done without touching the schematic.
(Looking up components, making footprints etc). Since the schematic was the biggest amount
of work, this produced a major bottle-neck for the schematic work. HOWEVER, having the overall
control of the entire thing in one document did help to smooth out quite a few issues we met
along the road, for instance we were having quite a few issues with netlabels, which were quite
easy to untangle when everything was in one document with 0 ports, as getting the complete overview
of where everything was, was quite easily readable with this approach. We also avoided needing
to use any buses this way (even if we did end up with one or two, for the sake of readability)
(TODO: Fill in a bit of details about pros for one-document vs multi). \\

The overall layout of the schematic is logically grouped "geographically", to
allow for easy reading of the schematic.

\subsection {Buses/Wirelabels}
We initially worked on the assumption that all pins should be connected to a bus, and then
that bus should be connected to the pins in the other end, this gave us noe end of issues with
duplicate naming, and after digging through quite a bit of Altium-documentation, we became
assured that simply wire-labeling the pins would create the necessary connections (as any
pin/wire with the same name as any other pin/wire will be connected by definition in Altium).

This arguably made for a less readable Schematic (as the bus-connected solution was quite a lot
easier to follow when tracing, however as our schematic still is logically grouped, it isn't very hard
to find the correct connections even without the busses drawn in. (Albeit with a bit of a learning curve
at the beginning, to learn where the various components are placed). We are quite sure that given
a logical overview of which components are directly connected to eachother, finding the various pins
that perform this connection should be trivial even without the busses/wires drawn in.

TODO: Overordnet forklaring av hvordan
