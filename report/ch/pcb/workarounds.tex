\section {Problems and Workarounds}
\subsection{Serial port}
\label{sec:serial}
We found many pin mappings for the serial port on the web, which turned out to
be different from Festina Lente's serial system. After some discussion, we chose
the one on the web, and figured that if anything did not work, we could just
use our header to change the mapping.  Yet as we had not thought of the fact
that the footprint for male and female serial ports have mirrored numbering,
we ended up with a male pin mapping and a female footprint.  Obviously,
this did not work very well. Yet as the TXD pin is the only one we actually
need, we used a female plug, and dragged a wire over the header pins so that
the TXD data from SCU were mapped to the original RXD. This workaround made it
possible to send data from the system and out to a computer, yet as we had no
need for this option, our final card does not have the serial system soldered on.

\subsection{Routing}
We spent the entire final week before production working on the routing. In comparison, the energy group used only 2 days. There are quite a few possible reasons why we ended up using so much more time than them for this:

One of the reasons were simply that we were the first group to start routing. We thus had no one to warn us about possible mistakes. An example of such a mistake was that we did not set the correct Design Rules before auto-routing, until a few days into the week. This naturally did not give us the routing that we wanted, and gave us a few headaches.

Other reasons might be the fact that we did not experiment enough with different routing strategies, and the fact the other group had to route only 1 memory chip, where we had 5. This naturally made the routing much more complex.

\subsection{Soldering}
Soldering was the biggest problem that we faced, as none of the members of the
\ac{PCB} group had any experience with it beforehand. We managed to
destroy our first board, as we used too much tin on one of the AVR pins, and
ruined them by using the tin removal unit wrongly.

However, the second attempt was more successful. But there still remained some
minor problems. Problems we found were lack of tin on the oscillator, as well as
a few of the \ac{FPGA} pins. They were, as opposed to our problem with the first
board, easily fixed, once discovered.

\subsection{The Third Board}
After using the second board for development for about a week, several problems started appearing.

First, we discovered that one of the connections between the AVR and FPGA was broken. We managed to work around this by replacing it with another previously unused connection.

Second, we discovered that one of the voltage regulators was a bit loose. Touching it could make the whole machine stop working.

If these had been the only problems, we could have avoided them by making sure not to touch the voltage regulator. We did however also discover a problem with the VGA memory. It would randomly stop working or misbehave, visible as noisy images on the screen.

We therefore decided that we had to make another board. To be able to do this we had to order some spare components. After the components arrived, we soldered the new board.

The testing of this board was done by running code on it. The programs that ran on the second board, also worked on the third board. The problem with the VGA memory was gone, and the voltage regulator seemed to be stable.
