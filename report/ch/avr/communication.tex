\section{Communication with LENA}
\label{sec:SCU-LENA-communication}

This section documents how the \ac{SCU} communicates and controls the
\ac{LENA}. The communication is usually performed by the \ac{SCU} assigning a
state to the \ac{LENA}, followed by some data transfer. The communication utilizes three buses: 
the LENA out bus, the LENA in bus, and the LENA state bus. A full pin reference of these buses can
be found in appendix \ref{app:lena-scu-bus}.

\subsection{LENA states}

The \ac{SCU} can set \ac{LENA} to 5 different states, controlling \ac{LENA}'s
behavior. The states are set with a 3-line bus and a fourth line which when
toggled, indicates that the \ac{LENA} should set its state to the state
currently on the bus. The three state lines represent a binary number referring
to one of the states. The different states are listed in table \ref{tab:states}.

\begin{table}[h]
  \centering
  \begin{tabular}{l l r} \toprule
    \thx{Name} & \thx{Description} & \thx{Comment}\\ \midrule
    \tt STOP & LENA stops & default state \\ 
    \tt RUN & LENA runs program \\ 
    \tt LOAD\_DATA & LENA receives data \\
    \tt STORE\_DATA & SCU receives data & not used \\
    \tt LOAD\_INSTRUCTION & LENA receives instructions\\
    \bottomrule
  \end{tabular}
  \caption{LENA states}
  \label{tab:scu-lena-states}
\end{table}


\subsection{Data transfer from SCU to LENA}
When the \ac{SCU} transfers data, it first sets the \ac{LENA} state to {\tt
  LOAD\_DATA} or {\tt LOAD\_INSTRUCTION} to load data or instructions,
respectively. The procedure for the \ac{SCU} is then to put one word on the bus,
toggle the data ready line and repeat. The \ac{LENA} state STOP is used to signal the
end of transfer. When the data are simply data, the first 8 lines of
the \ac{LENA} input bus is used to transfer data one byte at a time. When the
data are instructions for the \ac{LENA}, data are transferred one instruction (24
bit) at a time, and the first 24 lines of the bus are used.
%The entire procedure can be summarized by the following six steps:

%\begin{enumerate}
%\item SCU sets LENA state to LOAD\_DATA or LOAD\_INSTRUCTION
%\item SCU puts 8 or 24 bits of data on the bus
%\item SCU toggles the Address increment CLK line
%\item Repeat stage 2 through 3
%\item SCU sets LENA state to STOP
%\end{enumerate}

\subsection{Data transfer from LENA to SCU}
The \ac{LENA} has the ability to transfer data back to the \ac{SCU}. This is a
fallback routine, implemented in case of a bottleneck or failure in the \ac{VGA}
controller on the \ac{LENA}. In this case, the \ac{LENA} can transfer the
processed data back to the \ac{SCU} for storage or to be sent through an
alternative \ac{VGA} controller. In this transfer the data bus are 8 pins
connected to the \ac{LENA}'s \ac{I/O} pins.

This transfer is rather unusual because it is, unlike all other operations, not
initiated by the \ac{SCU}. The \ac{LENA} will initiate the transfer by putting
one byte on the bus, and then interrupt the \ac{SCU}. After the \ac{SCU} has
acknowledged, another byte can be transferred. The first four bytes of the
transfer will represent, as a 32 bit unsigned integer, the number of bytes of
actual data to be transferred.
%The entire procedure can be summarized by the following five steps, where \emph{x} is the amount of actual data.

%\begin{enumerate}
%\item LENA puts one byte of data on the bus
%\item LENA interrupts SCU (signaling there is data on the bus)
%\item SCU reads one byte of data from the bus
%\item When ready for more data, SCU toggles the Address increment CLK line
%\item Repeat stage 1 through 4 \emph{x} times
%\end{enumerate}
