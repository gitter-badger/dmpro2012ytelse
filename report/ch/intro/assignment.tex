\section{Assignment}
The task given was to create an {\em array-based} parallel image processor with
focus on performance. An array processor is a grid of processing elements, where
each of the processing elements are only able to communicate with its north,
south, east and west neighbours. All the processing elements will perform the
same instruction at all times, which means that the more processing elements one
have, the better performance one will get. From calculations we did, as shown in
Figure \ref{we-need-fig}, we found out that we were able to get enough
throughput to show a low quality unprocessed video, as a proof of concept.

Fast image processing is essential in robots\cite{miller1989r-vision,
  thrun2007stanley} and in artificial intelligence\cite{hwang1989parallel} in
order to work in the real world. Autonomous cars and robots need to process
images from image sensors fast enough to react and e.g. prevent
accidents\cite{aufrere2003coll-avoid}. Image processing in general is also
highly applicable in the medical field\cite{luong2009medical-image,
  sternberg1983biomedical} and in the petroleum
industry\cite{ferrari2007steam-images}.

\subsection{Original Assignment Text}

The original assignment was given as follows:

\begin{quotation}\em
The performance increase available from harvesting Instruction Level Parallelism
(ILP) from the serial instruction stream is limited because we have reached the
maximum power consumption that can be handled without expensive cooling
solutions \cite{olukotun2005future}. Consequently, there is a significant
interest in single-chip parallel processor solutions (e.g. \cite{bell2008tile64,
  kongetira2005niagara}).

The processor cores in commercial multi-core chips are conventional designs and
therefore reasonably complex. In this work, your task is to design an
array-based parallel image processor. An array processor is organized as a
matrix of processing elements where each element communicates with its neighbors
in the north, south, east and west directions.

Your image processor will be implemented on an FPGA, and you are free to choose
how to realize your array-based computer architecture. The system should be
shown to work with a suitable application. Studying the architecture of the
Goodyear MPP \cite{batcher1980design,wiki:goodyear} might be a possible starting
point.  Due to a large number of students this year, we will divide the work
into two independent projects: a) Performance and b) Energy efficiency. The goal
of group a) is to achieve maximum performance while group b) should try to
balance performance and energy. The reports from both groups should include an
evaluation of prototype performance and energy consumption.

\subsubsection*{Additional requirements}
The unit must utilize an Atmel AVR micro controller and a Xilinx
\acfoot{FPGA}. The budget is 10.000 NOK, which must cover components and
\acfoot{PCB} production. The unit design must adhere to the limits set by the
course staff at any given time. Deadlines are given in a separate time schedule.
\end{quotation}
