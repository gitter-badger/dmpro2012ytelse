\section{Script format}

The programs on the 256 shades of gray will at the top level be a script file containing information such as the FPGA binary program, the compatible data type (video, still picture, etc.) and speed of execution. The user will select a script, and will then get to choose a data unit of the compatible data type to run the program with.

The script is a text file containing four lines. The individual lines will contain the following information:

\begin{enumerate}
\item Program name (for the user interface)
\item Location of FPGA program binary file
\item FPGA program's compatible data type
\item Transmission delay of each file in ms (typically framerate)
\end{enumerate}

\subsection{Directory structure}

The different levels of the directory structure has its own use. The user will first choose a script from the first directory, and then a data unit from the second directory. The data at the root directory, the first level, is data associated with the script files. This level contains data types, fpga executables and the script files themselves. The data types are directories and the content of these directories represent the second level, the data units. A data unit is a collection of files viewed as a unit in the sense that a program can only be loaded with one unit at a time. A data unit can for instance be a still picture or an animation. Each data unit is also a directory, and their content represent the third level, the individual data unit files. When the selected data type is still pictures, the available data units will only contain one bitmap file. When, however, the data type is video, the data units will contain a set of bitmap files numbered after the order in which they should be loaded into the FPGA. %The file extention of the data unit files are '.lenna'.

The directory levels can be summarized accordingly:
\begin{enumerate}
\item \emph{Data type directories}, refered to as 'data types'.
\item \emph{Data unit directories}, refered to as 'data units'.
\item \emph{Data unit files}, all files that makes up a data unit.
\end{enumerate}
 
