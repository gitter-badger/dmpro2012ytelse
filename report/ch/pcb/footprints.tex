\section {Footprints}
Some of the components we chose did not have footprints readily available, which
meant we either had to look for them on the internet, or create some ourselves.

This usually meant either staring at datasheets and carefully placing pins
relative to each other, or running the IPC-wizard.

\subsection {We made the following footprints ourselves}
\begin{table}[h]
  \centering
  \begin{tabular}{l l}\toprule
    \thx{Component} & \thx{Purpose} \\ \midrule
    \ac{SD} card reader &  \\
    \ac{VGA} plug & For our own VGA implementation \\
    CY7C1069DV33 54TSOP & Data-memory (2M $\times$ 8bit) \\
    CY7C1021DV33 44TSOP & Program-memory (64K $\times$ 16bit) \\
    CY7C1049DV33 44TSOP &  VGA memory (512 $\times$ 8bit) \\
    \bottomrule
  \end{tabular}
  \caption{The Customized Footprints}
  \label{fig:footprints-we-made}
\end{table}

\begin {itemize}
\item SD-card 
  With the pin-designations selected by googling the pinouts for SD-cards in
  general, and comparing a few of those hits to make sure that the results
  agreed this\footnote{\url{http://pinouts.ru/Memory/sdcard_pinout.shtml}} 
  ended up being what we based the schematic-component based on. This 
  schematic\footnote{\url{http://katalog.we-online.de/em/datasheet/693063010911.pdf}}
  schematic was particularly hard to read, to the point that we had to ask 
  Tufte for help to understand it, and even after that, there was no way to
  be sure precisely where the grounding pads were relative to the rest of the 
  SD-card-reader. We thus ended up with a bit of guesswork, and added quite a 
  bit of slack to the ground-pads by simply adding extra tin to these pads.
\item Memory (TSOP54/TSOP44)
The TSOP54/TSOP44 footprints were created using the IPC Footprint Wizard, as
their datasheets fitted nicely with the Small Outline Package-setting in that
Wizard.
\end {itemize}

\subsection{Footprints from other sources:}
As Festina Lente used some components that we also ended up using, we decided,
after talking to Tufte, to use their Footprints\footnote{\url{http://org.ntnu.no/datamaskinerprosjekt2011/altium_libraries/dmprolibrary/}}
(as they were known to work) for these components:

\begin{table}[h]
  \centering
  \begin{tabular}{l l}\toprule
    \thx{Component} & \thx{Source} \\ \midrule 
    Button (FSM2JSMA) & Festina Lente \\
    Crystal & Festina Lente \\
    LUMBERG - 2486 01 - MINI \ac{USB} TYPE B & Festina Lente \\
    Oscillator (CSX750FJC50.000M-UT) & Festina Lente \\
    Power Connector & Festina Lente \\
    \ac{VGA} module & Festina Lente \\
    AVR & AVR-freaks \\
    \ac{FPGA} & Altium Design Library \\ \bottomrule
  \end{tabular}
  \caption{Footprints and sources}
  \label{fig:footprints-and-sources}
\end{table}

The oscillator had one issue last year, namely that the
footprint was mirrored, noting this from Festina Lente's report, we read
through the datasheet, and remapped the pin-numbering on the foot-print 
before using this footprint.

After reading the Festina Lente report\cite{berg2011festinalente}, we decided to
follow their choice of footprint for capacitors and resistors: 1206 SMT. As
those were physically bigger than the 0804 SMTs they mentioned as an
alternative, they would be easier to solder by hand.
