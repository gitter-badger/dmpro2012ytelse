\section{Requirements Specification}

\begin{table}[h]
  \centering
  \begin{tabularx}{\textwidth}{c X c}\toprule
    \thx{Name} & \thxc{Description} & \thx{Priority}\\ \midrule
    {\sc FR1} & \FRI   & {\sc High}   \\
    {\sc FR2} & \FRII  & {\sc High}   \\
    {\sc FR3} & \FRIII & {\sc High}   \\
    {\sc FR4} & \FRIV  & {\sc Medium} \\
    {\sc FR5} & \FRV   & {\sc Medium} \\
    {\sc FR6} & \FRVI  & {\sc Low}    \\
    {\sc FR7} & \FRVII & {\sc Low}    \\ 
    \bottomrule
  \end{tabularx}
  \caption[Functional requirements]{The functional requirements}
  \label{fig:func-req}
\end{table}

 A few functional
and most non-functional requirements were given to us by our instructor, Magnus
Jahre. The rest were decided by the group as goals we thought were realizable, and
some to help us see how far away we are from our goals. Table
\ref{fig:func-req}, which shows the functional requirements, includes a relative
priority between the different requirements.  This priority tells us what we
have focused on, as well as what is important in terms of success of the
computer. Clearly, focusing on performance, as specified in {\sc FR1}, is more
important than having developer tools for the machine ({\sc FR6}).

{\sc FR1} ensures a focus on performance. {\sc FR2} ensures that we do not end
up with a system which is not generally programmable. This is important, as a
computer specialized for image processing has less usability than a generally
programmable computer. {\sc FR3} and {\sc FR7} makes it easier to show that the
computer is capable of processing images, whereas {\sc FR4} through {\sc FR6} makes it
easier to use, debug and create programs for the computer.

\begin{table}[h]
  \centering
  \begin{tabularx}{\textwidth}{l X}\toprule
    \thxc{Name} & \thxc{Description}\\ \midrule
    {\sc NFR1} & The machine must use a Xilinx Spartan 3 XC3S500E PQG208 FPGA\\
    \midrule
    {\sc NFR2} & The machine must use one AVR32 UC3A microcontroller\\
    \midrule
    {\sc NFR3} & The budget of $\sim$ 10 000 NOK must cover all \ac{PCB} and
    component costs\\
    \midrule
    {\sc NFR4} & The image processor should consist of multiple cores arranged 
    in a matrix\\
    \midrule
    {\sc NFR5} & The machine should be optimized for performance\\
    \bottomrule
  \end{tabularx}
  \caption[Non-functional requirements]{The non-functional requirements}
  \label{fig:nonfunc-req}
\end{table}


Table \ref{fig:nonfunc-req} shows the non-functional requirements. These were
all determined by the assignment. They were all treated as absolute requirements,
and our design has been centered around those. {\sc NFR1} and {\sc NFR2} gave us
little choice in what hardware to use, and naturally this hardware have been used. {\sc
  NFR3} limits what components we can use to the cheaper ones. {\sc NFR4} Constraints our
design choices when designing the image processor. We could not make any type of core
organization, we had to make this specific type. {\sc NFR5} leads to some
design choices for the entire system. Our goal was to get the best possible performance, and we did not
have to care about concerns such as energy consumption. 
However, we may have sacrificed some simplicity in our design for
the sake of performance.
