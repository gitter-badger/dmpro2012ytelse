\section {Footprints}
Some of the components we chose did not have footprints readily available, which
meant we either had to look for them on the internet, or create some ourselves.

This usually meant either staring at datasheets and carefully placing pins
relative to each other, or running the IPC-wizard.

\subsection {We made the following footprints ourselves}
\TODO{Make the links to references}
\begin {itemize}
\item SD-card \url{http://katalog.we-online.de/em/datasheet/693063010911.pdf}
  With the pin-designations selected by googling the pinouts for SD-cards in
  general, and comparing a few of those hits to make sure that the results
  agreed \url{http://pinouts.ru/Memory/sdcard_pinout.shtml} ended up being what
  we based the schematic-component based on. This schematic was particularly hard
  to read, to the point that we had to ask Tufte for help to understand it, and even
  after that, there was no way to be sure precisely where the grounding pads were relative
  to the rest of the SD-card-reader. We thus ended up with a bit of guesswork, and added
  quite a bit of slack to the ground-pads by simply adding extra tin to these pads.
\item VGA-plug \url{http://www.te.com/commerce/DocumentDelivery/DDEController?Action=srchrtrv&DocNm=82068_AMPLIMITE_Right-Angle_Posted_Conn&DocType=Catalog+Section&DocLang=English&PartCntxt=}
\item Memory (TSOP54/TSOP44)
The TSOP54/TSOP44 footprints were created using the IPC Footprint Wizard, as
their datasheets fitted nicely with the Small Outline Package-setting in that
Wizard.

\begin {itemize}
\item Data-memory (2M x 8bit) CY7C1069DV33 54TSOP:
\url{http://www.cypress.com/?docID=31945} 
\item Program-memory (64K x 16bit)
CY7C1021DV33 44TSOP: \url{http://www.cypress.com/?docID=31965} 
\item VGA memory (512 x 8bit) CY7C1049DV33 TSOP44 : \url{http://www.farnell.com/datasheets/1468461.pdf}
\end {itemize}
\end {itemize}

\subsection{Footprints from Festina Lente:}
As Festina Lente used some components that we also ended up using, we decided,
after talking to Gunnar\CHECK{Should we refer by first or last name?}, to use
their Footprints (as they were known to work) for these components:
\url{http://org.ntnu.no/datamaskinerprosjekt2011/altium_libraries/dmprolibrary/}
\begin{itemize}
\item Button
\item Crystal
\item USB
\item Oscillator - The oscillator had one issue last year, namely that the
  footprint was mirrored, noting this from Festina Lente's report, we read
  through the datasheet (REFERENCE)\TODO{Reference}, and remapped the
  pin-numbering on the foot-print before using this footprint.
\item Power Connector
\item VGA-module
\end{itemize}

In addition we used the following premade footprints:
\begin{itemize}
\item AVR-footprint | from AVR-freaks.
\item FPGA-footprint | Built-in from the Altium Designer Library.
\end{itemize}

After reading the Festina Lente-report\TODO{Reference}, we decided to follow their
choice of footprint for capacitors and resistors, 1206 SMT, as those were physically bigger
than the 0804 SMTs they mentioned as an alternative, and thus would be easier to solder by hand.
