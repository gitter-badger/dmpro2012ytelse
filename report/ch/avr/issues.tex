\section{Issues}
While we did not have any serious problems with the SCU, we did
encounter a few. This section is intended to shed some light on the
issued we faced and how we were able to either fix or get around them. 

\subsection{SPI}
\label{sec:avr-spi-issues}
Reading from the \ac{SD} card over \ac{SPI} turned out to be the single biggest
bottleneck of our system. This was in part due to the serial nature of \ac{SPI},
but also the poorly optimized \ac{SPI} driver from the Atmel Software Framework.

The limitations of \ac{SPI} is not something we could change. Our
microcontroller does not have hardware support for using the 4-bit \ac{SD}
protocol, which left us with \ac{SPI} as our only choice. We ended up running
the \ac{SD} card at a frequency of 39 MHz, which gives us a theoretical upper
speed bound of 4.88 MB/s. In practice, we managed to reach around 1.1 MB/s.

Some of the changes we made to make this happen, was: \vspace{-1.0em}
\begin{itemize}
  \item Turn on compiler optimizations \vspace{-1.0em}
  \item add SDHC support and increase clock rates\vspace{-1.0em}
  \item use multiple block reads and bypass the file system
\end{itemize}

This is explained in more detail in Section \ref{sec:performance-sd-card}.

\subsection{Serial}
\label{sec:avr-serial-issues}
Communicating with the AVR over the serial port worked perfectly fine on the
EVK1100. However, on 256 Shades of Gray it did not work at once. The
only obvious error with the serial device was that the TXD/CTS (T for Transmit)
pins were wired into RXD/RTS (R for Receive), and vice versa. This is covered 
in more detail later on, in the PCB workaround chapter. However, 
as the Str2img was completed, debugging could be done by writing text to screen.
As debugging was the main purpose of the serial communication, we did not prioritize
fixing the serial. In the end, we only made it work to convince ourselves 
that our hypothesis was correct.

\subsection{Dead pin between SCU and LENA}
One of the unidirectional pins between SCU and LENA turned out to dead.
This was simple enough to work around by replacing it with one of the
bidirectional pins.

\subsection{Bug in RTC Driver}
When initializing the Real Time Clock (RTC) driver, we encountered a
known
bug\footnote{\href{http://asf.atmel.com/bugzilla/show_bug.cgi?id=10}{Atmel
Bug Report}}. This meant that we could not use the internal 32kHz
crystal oscillator for measuring time, but had to use the 115kHz RC
oscillator instead. This meant slightly higher inaccuracies, but for our
purposes, it was not a a serious issue.
