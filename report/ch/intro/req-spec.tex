\section{Requirements Specification}
\begin{comment}
\begin{table}[h]
  \centering
  \begin{tabularx}{\textwidth}{l X}\toprule
    \thxc{Name} & \thxc{Description}\\ \midrule
    {\sc NFR1} & The machine must use a Xilinx Spartan 3 XC3S500E PQG208 FPGA\\
    \midrule
    {\sc NFR2} & The machine must use one AVR32 UC3A microcontroller\\
    \midrule
    {\sc NFR3} & The budget of $\sim$ 10 000 NOK must cover all \ac{PCB} and
    component costs\\
    \midrule
    {\sc NFR4} & The image processor should consist of multiple cores arranged 
    in a matrix\\
    \midrule
    {\sc NFR5} & The machine should be optimized for performance\\
    \bottomrule
  \end{tabularx}
  \caption[Non-functional requirements]{The non-functional requirements}
  \label{fig:nonfunc-req}
\end{table}

 % Vi are supposed to write about the (non)functional requirements of the
 % project. Not about (non)functional requirements and their importance
 % in general, the reader already knows what these are and their importance.
Creating a computer without specifying functional requirements would make it
difficult to both see how well we are progressing and to figure out if we've
actually reached our goal in the end. Non-functional requirements are important
to realize the computer: Both price and components were set by the assignment as
non-functional requirements. As such, both functional and non-functional
requirements were written down to give ourselves clear goals and
specifications. 

\begin{table}[h]
  \centering
  \begin{tabularx}{\textwidth}{c X c}\toprule
    \thx{Name} & \thxc{Description} & \thx{Priority}\\ \midrule
    {\sc FR1} & \FRI   & {\sc High}   \\
    {\sc FR2} & \FRII  & {\sc High}   \\
    {\sc FR3} & \FRIII & {\sc High}   \\
    {\sc FR4} & \FRIV  & {\sc Medium} \\
    {\sc FR5} & \FRV   & {\sc Medium} \\
    {\sc FR6} & \FRVI  & {\sc Low}    \\
    {\sc FR7} & \FRVII & {\sc Low}    \\ 
    \bottomrule
  \end{tabularx}
  \caption[Functional requirements]{The functional requirements}
  \label{fig:func-req}
\end{table}



Table \ref{fig:func-req}, which shows the functional requirements includes a
relative priority between the different requirements. This priority tells us
what we have focused on, as well as what is important in terms of success of the
computer. Clearly, focusing on performance, as specified in {\sc FR1}, is more
important than having developer tools for the machine ({\sc FR6}).
\CHECK{Ensure FR numbers are correct.}

{\sc FR1} ensure the focus on performance. {\sc FR2} ensures that we do not end
up with a system which is not generally programmable. This is important, as a
computer specialized for image processing has less usability than a generally
programmable computer. {\sc FR3} and {\sc FR7} makes it easier to show that the
computer is capable of processing images, whereas {\sc FR4 - FR6} makes it
easier to use, debug and create programs for the computer.
\end{comment}
\begin{table}[h]
  \centering
  \begin{tabularx}{\textwidth}{c X c}\toprule
    \thx{Name} & \thxc{Description} & \thx{Priority}\\ \midrule
    {\sc FR1} & \FRI   & {\sc High}   \\
    {\sc FR2} & \FRII  & {\sc High}   \\
    {\sc FR3} & \FRIII & {\sc High}   \\
    {\sc FR4} & \FRIV  & {\sc Medium} \\
    {\sc FR5} & \FRV   & {\sc Medium} \\
    {\sc FR6} & \FRVI  & {\sc Low}    \\
    {\sc FR7} & \FRVII & {\sc Low}    \\ 
    \bottomrule
  \end{tabularx}
  \caption[Functional requirements]{The functional requirements}
  \label{fig:func-req}
\end{table}


Several functional and Non-functional requirements were given to us by our
instructor, Magnus Jahre, and some were decided by the group as goals we
though realizable or just to help us see the progress. Table \ref{fig:func-req}, which shows the functional
requirements, includes a relative priority between the different requirements.
This priority tells us what we have focused on, as well as what is important in
terms of success of the computer. Clearly, focusing on performance, as
specified in {\sc FR1}, is more important than having developer tools for the machine ({\sc FR6}).
\CHECK{Ensure FR numbers are correct.}

{\sc FR1} ensure the focus on performance. {\sc FR2} ensures that we do not end
up with a system which is not generally programmable. This is important, as a
computer specialized for image processing has less usability than a generally
programmable computer. {\sc FR3} and {\sc FR7} makes it easier to show that the
computer is capable of processing images, whereas {\sc FR4 - FR6} makes it
easier to use, debug and create programs for the computer.

\begin{table}[h]
  \centering
  \begin{tabularx}{\textwidth}{l X}\toprule
    \thxc{Name} & \thxc{Description}\\ \midrule
    {\sc NFR1} & The machine must use a Xilinx Spartan 3 XC3S500E PQG208 FPGA\\
    \midrule
    {\sc NFR2} & The machine must use one AVR32 UC3A microcontroller\\
    \midrule
    {\sc NFR3} & The budget of $\sim$ 10 000 NOK must cover all \ac{PCB} and
    component costs\\
    \midrule
    {\sc NFR4} & The image processor should consist of multiple cores arranged 
    in a matrix\\
    \midrule
    {\sc NFR5} & The machine should be optimized for performance\\
    \bottomrule
  \end{tabularx}
  \caption[Non-functional requirements]{The non-functional requirements}
  \label{fig:nonfunc-req}
\end{table}


Table \ref{fig:nonfunc-req} shows the non-functional requirements. These were
all determined by Jahre. They are all treated as absolute requirements, and our
design has been centered around these requirements. {\sc NFR1} and {\sc NFR2} gave us little
choice in what hardware to use, and naturally these have been used. {\sc NFR3} limits
what peripherals we can use to relatively cheap things. {\sc NFR4} leads to some design
choices when designing the image processor, we can't just make any set of SIMD cores, we have
to make this specific type!  {\sc NFR5} leads to some design choices for the entire system.
Our goal is performance, and we don't have to care about other concerns such as energy
consumption, just to get performance as good as possible without the system burning up.
\ref{fig:nonfunc-req}
