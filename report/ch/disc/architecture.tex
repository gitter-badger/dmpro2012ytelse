\section {Architectural design}
\TODO{Graph of memory-architechture}
In the early phases of the project, multiple solutions as to how the data memory should work were suggested. 
The reason we decided to let the AVR send all the data to the FPGA, and then let it store them to its own local memory, 
was due to all the possible issues with shared memory. A shared memory would easily be prone to errors from synchronization 
problems such as knowing when, and to which addresses, a read/write operation should read/write. 
Instead, we figured that we would rather send all data from the AVR to the FPGA, and let it store them into its own memory as it saw fit.

The program memory is separated on two chips, as we found no good 24 bit word memory, and it was more expensive with one 32 bit memory, 
than our alternative solution, which combined two 16 bit ones. \TODO{See if we really need to say this in the PCB-chapter too}

\TODO {find out exactly what was the issue here, with the SD card, frame size, and on-chip AVR memory}
As we got deeper into the techincal parts of our system, we figured we needed two additional memories. 
One for the AVR, as it had little on-chip memory, and thus might get some issues as it reads data from the SD card in blocks.\CHECK{(?)} 
Another argument for adding this memory, was for our backup solution to work, as the FPGA would then have to send data back to the AVR, and the AVR would then have to store it somewhere.

Also, as a shared VGA and data memory might obstruct the flow of data from the AVR. 
As the data for the VGA would be both read from and written to a lot, the VGA needed it's own memory, 
rather than share space with the data. As we were supposed to optimize for performance, we could not 
afford to have this as a potentional bottleneck, and thus invested in an extra memory chip.

We feel that both of these additional memory components were necessary, even though we might have done 
without the AVR memory. Although, as not having it would significantly lowered the preformance if 
the VGA module had to be used, we feel it was well worth the extra cost. \TODO{This sentence simply restates what is said in the first sentence of this paragraph}

\TODO {Find out if serial works}
As both the earlier groups had done so, and because Jahre told us to, we had multiple redundancies in order 
to safeguard against possible errors. In order to send data into the system, our two main options would be 
to use the SD-card or mini-USB plug. For debugging purposes, we had serial and JTAG, and for the output we 
had the two VGA options. Also, we used headers so as to be able to alter our pin usage, if need be. 
Luckily, both SD card and and our own VGA worked, and we therefore had no need of the redundancies. \TODO{Motivate the need for the redundancies as well, instead of just
saying they were unneccesary}
