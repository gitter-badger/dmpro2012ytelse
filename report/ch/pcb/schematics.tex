\subsection {Schematics}

We decided to design the entire \ac{PCB} in one schematic, as the Festina Lente
report mentioned that Festina Lente-report does mention that 
``The decision to make the central components appear in multiple schematic
sheets made Altium issue a lot of warnings and errors during the design rule 
check''\cite[p.~49]{berg2011festinalente}. Something we avoided from day one, 
as we never attempted to use multiple schematics-documents.

A major downside of this approach was the fact that this completely serialized
our work on the Schematic, since we could not make concurrent changes to our
single document. This limited the parallelism of our work flow to one person at a
time working on the schematic.  In the meantime, the other people in the
\ac{PCB} group did whatever could be done without touching the
schematic. (Making footprints, verifying design, looking up parts and data
sheets) Since the schematic was the biggest amount of work, this produced
something of a bottleneck for the schematic work.

However, having the overall control of the entire thing in one document
did help to smooth out some issues we met along the road. For instance,
we were having quite a few issues with net labels. This was quite easy to
untangle when everything was in one document with no ports, as getting the
complete overview was very readable with this approach. We also avoided the need
to use any buses, although we did end up with one or two for the sake
of readability.

The overall layout of the Schematic \ref{app:schematics} is logically grouped ``geographically'', to
allow for easy reading of the schematic.

\subsubsection {Buses/Wirelabels}
We initially worked from the assumption that all pins should be connected to a
bus, and then that bus should be connected to the pins in the other end. This
gave us some issues with duplicate naming. After digging through quite a bit of
Altium documentation, we became assured that simply wire-labeling the pins
would create the necessary connections. This is because any pin/wire with the
same name as any other pin/wire will be connected by definition in Altium.

This arguably made for a less readable schematic, as the bus-connected solution
was quite a lot easier to follow when tracing. However, as our schematic still
is logically grouped, it is not very hard to find the correct connections even
without the buses drawn in. We are quite sure that given a logical overview of
which components are directly connected to each other, finding the various pins
that perform this connection should be trivial even without the buses/wires
drawn in.

\TODO{Overordnet forklaring av hvordan (JN: ???)}
