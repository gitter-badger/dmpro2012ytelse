\section{VGA-controller}

\TODO{Sort of a braindump, feel free to prune or rewrite whatever.}

\subsection{Motivation}
\TODO{Should this be a subsection at all?} We learned early on that last years
group had problems with their off-the-shelf VGA-controller being a major
bottle-neck, and were unable to squeeze out more than sub one frame per second
\TODO{Rephrase?}. Since one of our goals \TODO{Which goal?} in this project was
to hit a minimum frame rate of 10 per second, going down the same route with the
same controller was obviously off the table.

After having scoured the net for higher-performing VGA-controllers in our
price-range without any luck, and after consulting with Jahre, we decided on
implementing our own VGA-controller within the FPGA.

\subsection{The design}
\TODO{Insert conceptual block-figure of controller.}
\subsubsection{FPGA design}

\subsubsubsection{Memory handling}
We recognized the importance of allowing the control core to dump pixels to the
VGA-controller at its own pace without having to meet certain timing criteria,
so a physical memory and hence a memory controller was needed.  Since the FPGA
is running on 50MHz and the pixel clock should be running on approximately
25MHz, a fairly simple solution for dividing the memory between the control core
and the actual pixel pusher\TODO{Find a more serious name?} was found: The pixel
pusher must be able to read from memory at most every other cycle, so the
remaining cycles are all free for the control core to use.  To simplify the
design as much as possible, the memory controller simply alternates every cycle
between writing whatever is asserted on the signals from the control core and
reading a pixel for the pixel pusher.

\subsubsubsection{Pixel pusher} \TODO{Again, better name?}
The part that actually yields the output is pretty straightforward. It simply
calculates when to pulse the V-sync and H-sync signals for a given resolution,
and outputs a (8-bit greyscale) pixel fetched from the memory controller at the
appropriate time.
\TODO{There isn't much more to it. Get more technical?}

\subsubsection{Circuitry}
\TODO{Insert small circuit diagram of the physical parts?}
When designing the circuit we had to choose between a black-box\TODO{Call it
something else?} DAC and rolling our own. Preliminary research on the
VGA-protocol showed that for our purposes and fairly low requirements on
quality, making the DAC with simple resistors in parrallel, with each resistor
doubling in resistance for each step from the most significant bit to the least
significant, would do just fine. We thought this would be easier to debug, and
especially easier to prototype.

We ended up supporting greyscale and color with our circuit, but for now only
greyscale is implemented on the FPGA-side.
