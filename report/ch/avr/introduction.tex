The \acf{SCU} is implemented on an AT32UC3A0512 AVR microcontroller. This was both an advantage and a disadvantage to us. The same AVR had been used in several previous projects which allowed us to learn from their success and failures to a greater degree. At the same time this specific AVR gave us some issues, especially in regard to performance on the \ac{SD} card.

As the name implies, the \ac{SCU} is the controlling unit in
our system. The \ac{SCU} is in charge of setting up both \ac{LENA}'s instruction
and data memory. \ac{LENA} Starts when it is set to a run state by the
\ac{SCU}. After the processing is done the \ac{LENA} will let the \ac{SCU} know
that the work has been completed by setting an interrupt bit. It's then up to
the \ac{SCU} to decide what to do next.

The \ac{SCU} also implements the user interface for the project. With the help
of the \ac{VGA} implementation on the \ac{LENA} it is able to display a menu to
the user which allows for selection of a program and data set from the \ac{SD}
card connected to the \ac{SCU}.
