\section {Introduction}

Design choices: \\
\subsection {Memory}
The overall design for our machine calls for quite a bit of different memories. One of the
earliest design choices that led to this was the decision to have separate instruction/data-memories.
The reasoning behind this choice being that we could avoid the bottleneck that would be introduced
from sharing memories. Since the requirements for these two memories differed in both size and word-width 
we wound up with not only separate, but also different chips for this purpose (The data-memory required
8-bit words, the instruction-width was 24-bit, and since we wanted to avoid using multiple memory accesses
to get a complete instruction, we needed a wide enough memory chip for that purpose. 24-bit chips were out of
production, and 32-bit memory was too expensive, thus the solution became 2x16-bit chips with their address-lines
connected together and 8 ignored I/O-pins, effectively making them a 24-bit memory).

Since we wanted to reduce the sharing of memories as much as possible, we also needed a separate memory for
our VGA-controller, as that needed to read it's buffer as fast as possible without interfering with the speed
of the rest of the system. This called for a memory that was big enough to hold atleast a full screen-frame,
at 8-bit per pixel (since each pixel is an 8-bit greyscale pixel).

To reduce the possibility of having too slow data-access from the AVR, an extra memory was added to work as
a buffer for the AVR as well. This design choice was made AFTER ordering, which meant that we had to choose from
the chips we had already ordered to fit this purpose. Since this was intended to carry data intended for the rest
of the system, and as the rest of the system is working with data in 8-bit bytes, we ended up using one of the
extra chips ordered as VGA-memory for this purpose.

\subsection{VGA}
Our initial plans were to create our own VGA-controller in the FPGA, but since having video-output was mission-critical
for our project, we really wanted to have a fallback-solution in case our own VGA-controller-solution ended up not working.
To solve this, we added in the necessary connectors for the VGA-module that Festina Lente used last year.

\subsection {Communication}
We planned on using the SD-Card-reader as our main source of data/instructions, however, as our system would
be completely useless unless we were able to get data into it at decent speeds, we opted to also have a fallback-solution
here, thus we also added USB and RS232 as fallback-solutions for getting data/communicating with the computer.

