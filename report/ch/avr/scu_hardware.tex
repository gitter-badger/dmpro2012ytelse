\section{SCU hardware}

This section documents the hardware considerations we had to make in relation
to the the microcontroller used in the SCU, the AVR AT32UC3A0, hereby referred
to as AVR. Also included are specifics in regard to what choices were made in
terms of AVR specific hardware. 

Trying to design and plan some of the hardware associated with the AVR, it was
plain that while some components could be connected to arbitrary GPIO pins,
others would require specific pins. To mention some of these:\TODO{only mention some, not all?}

\begin{enumerate}
\item Serial port
\item \ac{SD} card
\item \ac{USB} connectivity
\item \ac{EBI} AVR memory
\end{enumerate}

We also wanted to have some visual input and debugging tools for the project
(buttons and LEDs). These were also connected to the AVR. We did however not
have to take that into account mapping pins as they could be fitted on any
\ac{GPIO} pin. This allowed us greater flexibility in terms of what had to go
where and made it possible to fit the other interfaces on the AVR better.

Same goes for the \ac{VGA} controller we had mapped up to the AVR. Seeing
as this didn't require any specific ports on the AVR, we fit this in where
there were leftover pins when the rest was mapped out.

For the actual pinout of the AVR, see Appendix \ref{app:schematics}.

\begin{description}

\item[AVR crystal] \hfill

Previous projects have chosen to use multiple crystals on their AVR. Our AVR has support for up to 3 crystals, where 2 are high frequency crystals that can be used as external clocks for the AVR. The last crystal is a low frequency crystal that can replace an internal clock for power saving functions and real time measurement for greater accuracy, since the internal clock speed may vary by temperature.

As we were to optimize for performance we tried to use the most of the given resources. This included saving pins wherever we could, and as such we have gone with only one crystal. This crystal is connected to port 0 on the AVR and has a socket. We ordered a number of crystals in order to make sure that we could adapt to most any requirements that may or may not have been clear to us at the time of the design.

\item[SD card]  \hfill % \ac inside sections messes up TOC

We also needed a storage medium. Since previous projects have had good results with
using \ac{SD} cards, this was what we chose as well. However, working on the
design we encountered a few problems with this, mostly in
regard to performance. This problem is elaborated in Section
\ref{sec:performance-sd-card}.

\item[Serial port]  \hfill

The serial feature was implemented as a way of debugging. This is connected to
\ac{USART} 0 on the AVR and was not a crucial part of the design, but rather a
way for us to communicate and get quantitative data from the \ac{SCU}. We
decided that having serial access would be handy as there is only so much
information you can get out of 8 blinking \acp{LED}. The serial had some
implementation issues in regard to our mapping of the lines to the port. Since
we had the ability to debug from the \ac{VGA} controller, we did not prioritize
fixing the serial until later. In the end, we never had to use it, more than to
test whether it worked or not.

\item[USB connectivity]  \hfill 

Our AVR supports \ac{USB}, but it was not implemented in the \ac{SCU}. As both
the serial and the \ac{SD} card reader was supposedly easier to
implement\cite{berg2011festinalente}, we decided to defer the implementation
until other \ac{I/O} components were implemented. This feature was therefore
given a low priority and was not implemented in the final design due to time
constraints.
 
\item[External Bus Interface - AVR Memory]  \hfill 

Early in the design process, we came to doubt whether the \ac{SD} card would perform as well as we needed it to. In order to try to safeguard us from total failure in a case where this would come true (which it did) we gave the AVR a separate memory in order to be able to have a buffer for data coming from the \ac{SD} card or from the \ac{FPGA}. The AVR itself does not have enough memory for even a single frame of video, which is why -- after some discussion -- we deemed it a worthwhile use of ports at the expense of some \ac{I/O} lines to the FPGA.

\item[$\mu$VGA-II]  \hfill

Questioning whether or not we would be able to implement a \ac{VGA} controller
on the \ac{LENA}, we thought it was wise to have a backup in case this would fail. The $\mu$VGA-II is a prebuilt type that
has been used in previous projects with some success. In the backup solution, \ac{LENA} 
would pipe data back to the AVR for display on screen, should
the \ac{LENA} \ac{VGA} controller fail.

\end{description}
