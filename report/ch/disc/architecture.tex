\section {Architectural design}
\TODO{Graph of memory-architecture} In the early phases of the project,
multiple solutions as to how the data memory should work were suggested. We
decided to let the AVR send all the data to the \ac{FPGA}, and then let it store
them to its own local memory. The reason we chose this over shared memory was
all the potential issues one may get with such a solution. A shared memory would
easily be prone to errors from synchronization problems such as knowing when,
and to which addresses, a read/write operation should read/write. We figured
that we would rather send all data from the AVR to the \ac{FPGA}, and let it
store them into its own memory as it saw fit.

The program memory\CHECK{Program memory separated? This is news to me.} is
separated on two chips. The reason for this is mainly that we found no good 24
bit word memory. One solution would be to just buy a 32-bit memory. However, it
was more expensive with a single 32 bit memory, than two 16 bit ones. As both
gave us the needed memory, we went for the cheaper option.
\TODO{See if we really need to say this in the PCB-chapter too}

\TODO {find out exactly what was the issue here, with the SD card, frame size,
  and on-chip AVR memory} As we got deeper into the technical parts of our
system, we figured we needed two additional memories. One for the AVR, as it had
little on-chip memory, and thus might get some issues as it reads data from the
\ac{SD} card in blocks.\CHECK{(?)} Another argument for adding this memory was
to ensure our backup solution would work. As the \ac{FPGA} then would have to
send data back to the AVR, the AVR would have to store it somewhere.

In addition would a shared \ac{VGA} and data memory possibly obstruct the flow
of data from the AVR. As data for the \ac{VGA} would be both read and written
frequently, the \ac{VGA} needed its own memory, rather than sharing memory with
the AVR itself. As we were supposed to optimize for performance, we had to take
possible bottlenecks into consideration. We therefore invested in an extra
memory chip.

\CHECK{Find out if serial works} We had multiple redundancies to avoid potential
errors which would have rendered our whole system unusable. This was recommended
by Jahre, and previous groups had done so as well. In order to send data into
the system, our two main options would be to use the \ac{SD} card or the mini
\ac{USB} plug. We had serial and \ac{JTAG} for debugging purposes. For the
output we had the two \ac{VGA} options. Also, we used headers so as to be able
to alter our pin usage, if need be.\CHECK{What has headers to do with
  redundancy?  It's just convenience.} Luckily, both \ac{SD} card and and our
own \ac{VGA} worked, and we therefore had no need of the
redundancies. \TODO{Motivate the need for the redundancies as well, instead of
  just saying they were unnecessary}
